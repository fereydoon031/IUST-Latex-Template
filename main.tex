%  اگر مایلید پایان‌نامه شما دورو باشد به جای oneside در خط زیر از twoside استفاده کنید.
\documentclass[oneside,openany,msc]{IUST-Thesis}
 % در این فایل، دستورها و تنظیمات مورد نیاز، آورده شده است.
%-------------------------------------------------------------------------------------------------------------------

% در ورژن جدید زی‌پرشین برای تایپ متن‌های ریاضی، این سه بسته، حتماً باید فراخوانی شود
\usepackage{amsthm,amssymb}
%گزینه [fleqn] برای چپ چین کردن تمام فرمولها است. (برای وسط چین بودن آن‌ها، آکولاد را حذف کنید.)
\usepackage[fleqn]{amsmath}
% بسته‌ای برای تنطیم حاشیه‌های بالا، پایین، چپ و راست صفحه
\usepackage[top=30mm, bottom=30mm, left=25mm, right=40mm]{geometry}
% بسته‌‌ای برای ظاهر شدن شکل‌ها و تصاویر متن
\usepackage{graphicx}
\usepackage{pgfplots}
\usepackage{tikz}
\usetikzlibrary{calc,intersections}
% بسته‌ای برای رسم کادر
\usepackage{framed} 
% بسته‌‌ای برای چاپ شدن خودکار تعداد صفحات در صفحه «معرفی پایان‌نامه»
\usepackage{lastpage}
% بسته‌ و دستوراتی برای ایجاد لینک‌های رنگی با امکان جهش
\usepackage[pagebackref=false,colorlinks,linkcolor=blue,citecolor=blue]{hyperref}
% چنانچه قصد پرینت گرفتن نوشته خود را دارید، خط بالا را غیرفعال و  از دستور زیر استفاده کنید چون در صورت استفاده از دستور زیر‌‌، 
% لینک‌ها به رنگ سیاه ظاهر خواهند شد که برای پرینت گرفتن، مناسب‌تر است
\usepackage{xcolor}
\hypersetup{
	colorlinks,
	linkcolor={black},
	citecolor={black},
	urlcolor={black}
}
% بسته‌ای برای تنظیم نحوه ظاهر شدن اولین صفحه فصلها
\usepackage{titlesec}
% بسته‌ لازم برای تنظیم سربرگ‌ها
\usepackage{fancyhdr}
% تنظیم سر برگ دو خطی (مخصوص دانشگاه علم و صنعت)
\renewcommand\headrule{%
  \begingroup
  \vspace{5pt}
  \hrule height 2pt width\headwidth
  \vspace{1pt}
  \hrule height 0.7pt width\headwidth
  \endgroup
}

\def\MyHeader{\titleformat{\chapter}[display]{\centering \normalfont\huge\bfseries}
{ {\chaptertitlename} { \thechapter:}}{20pt}{\Huge}[\newpage \thispagestyle{fancy}]}
\def\MyBibHeader{\titleformat{\chapter}[display]{\centering \normalfont\huge\bfseries}
{ {\chaptertitlename} { \thechapter:}}{20pt}{\Huge}[\newpage \thispagestyle{plain}]}
\def\onvan{عنوان}

\def\MATtextbaseline{1.5}
\renewcommand{\baselinestretch}{\MATtextbaseline}
%
\usepackage{setspace}
\usepackage{algorithm}
\usepackage{algorithmic}
\usepackage{subfigure}
\usepackage[subfigure]{tocloft}
% بسته‌ای برای ظاهر شدن «مراجع» و «نمایه» در فهرست مطالب
\usepackage[nottoc]{tocbibind}
\setlength{\cftfignumwidth}{12mm}
% دستورات مربوط به ایجاد نمایه
\usepackage{makeidx}
\makeindex
% فراخوانی بسته زی‌پرشین و تعریف قلم فارسی و انگلیسی
% در صورتی که میخاهید زی‌پرشین تصمیم بگیرد که حروف را کشیده کند گزینه [Kashida] را به زی‌پرشین اضافه کنید.
\usepackage{xepersian}
\settextfont[Scale=1]{HM XZar}
\setlatintextfont[Scale=0.9]{Times New Roman}

%%%%%%%%%%%%%%%%%%%%%%%%%%
% چنانچه می‌خواهید اعداد در فرمول‌ها، انگلیسی باشد، خط زیر را غیرفعال کنید. (توجه داشته باشید که فونت‌های زیر بر روی رایانه شما نصب شده باشد.)
%\setdigitfont[Scale=1]{Persian Modern}%{XB Zar}
%%%%%%%%%%%%%%%%%%%%%%%%%%
% تعریف قلم‌های فارسی و انگلیسی اضافی برای استفاده در بعضی از قسمت‌های متن
\defpersianfont\titlefont[Scale=1]{HM XTitr}
% \defpersianfont\iranic[Scale=1.1]{XB Zar Oblique}%Italic}%
% \defpersianfont\nastaliq[Scale=1.2]{IranNastaliq}

%%%%%%%%%%%%%%%%%%%%%%%%%%
% دستوری برای حذف کلمه «چکیده»
\renewcommand{\abstractname}{}
% دستوری برای حذف کلمه «abstract»
%\renewcommand{\latinabstract}{}
% دستوری برای تغییر نام کلمه «اثبات» به «برهان»
\renewcommand\proofname{\textbf{برهان}}
% دستوری برای تغییر نام کلمه «کتاب‌نامه» به «مراجع»
\renewcommand{\bibname}{مراجع}
% دستوری برای تعریف واژه‌نامه انگلیسی به فارسی
\newcommand\persiangloss[2]{#1\dotfill\lr{#2}\\}
% دستوری برای تعریف واژه‌نامه فارسی به انگلیسی 
\newcommand\englishgloss[2]{#2\dotfill\lr{#1}\\}
% تعریف دستور جدید «\پ» برای خلاصه‌نویسی جهت نوشتن عبارت «پروژه/پایان‌نامه/رساله»
\newcommand{\پ}{پروژه/پایان‌نامه/رساله }

%\newcommand\BackSlash{\char`\\}

%%%%%%%%%%%%%%%%%%%%%%%%%%

\SepMark{-}
\setlength{\cftsubsecnumwidth}{4em}
\setlength{\cftsecnumwidth}{3em}
% تعریف و نحوه ظاهر شدن عنوان قضیه‌ها، تعریف‌ها، مثال‌ها و ...
\theoremstyle{definition}
\newtheorem{definition}{تعریف}[section]
\theoremstyle{theorem}
\newtheorem{theorem}[definition]{قضیه}
\newtheorem{lemma}[definition]{لم}
\newtheorem{proposition}[definition]{گزاره}
\newtheorem{corollary}[definition]{نتیجه}
\newtheorem{remark}[definition]{ملاحظه}
\theoremstyle{definition}
\newtheorem{example}[definition]{مثال}

%\renewcommand{\theequation}{\thechapter-\arabic{equation}}
%\def\bibname{مراجع}
\numberwithin{algorithm}{chapter}
\def\listalgorithmname{فهرست الگوریتم‌ها}
\def\listfigurename{فهرست تصاویر}
\def\listtablename{فهرست جداول}

%%%%%%%%%%%%%%%%%%%%%%%%%%%%

%\doublespacing
%%%%%%%%%%%%%%%%%%%%%%%%%%%%%
% دستوراتی برای اضافه کردن کلمه «فصل» در فهرست مطالب

\newlength\mylenprt
\newlength\mylenchp
\newlength\mylenapp

\renewcommand\cftpartpresnum{\partname~}
\renewcommand\cftchappresnum{\chaptername~}
\renewcommand\cftchapaftersnum{:}

\settowidth\mylenprt{\cftpartfont\cftpartpresnum\cftpartaftersnum}
\settowidth\mylenchp{\cftchapfont\cftchappresnum\cftchapaftersnum}
\settowidth\mylenapp{\cftchapfont\appendixname~\cftchapaftersnum}
\addtolength\mylenprt{\cftpartnumwidth}
\addtolength\mylenchp{\cftchapnumwidth}
\addtolength\mylenapp{\cftchapnumwidth}

\setlength\cftpartnumwidth{\mylenprt}
\setlength\cftchapnumwidth{\mylenchp}	

\makeatletter
{\def\thebibliography#1{\chapter*{\refname\@mkboth
   {\uppercase{\refname}}{\uppercase{\refname}}}\list
   {[\arabic{enumi}]}{\settowidth\labelwidth{[#1]}
   \rightmargin\labelwidth
   \advance\rightmargin\labelsep
   \advance\rightmargin\bibindent
   \itemindent -\bibindent

   \listparindent \itemindent
   \parsep \z@
   \usecounter{enumi}}
   \def\newblock{}
   \sloppy
   \sfcode`\.=1000\relax}}
\makeatother

% تنظیم سفارشی سربرگ
\pagestyle{fancy}
\fancyhf{}
 \fancyhead[L]{ \leftmark}
 \fancyhead[R]{\Onva}
 \fancyfoot[C]{\thepage}
 \renewcommand{\chaptermark}[1]{%
\markboth{\ #1}{}}

% بسته ای برای ریست کردن شماره پاورقی در هر صفحه
\usepackage{perpage} %the perpage package
\MakePerPage{footnote}
%دستوراتی برای پاورقی (شماره گذاری انگلیسی برای پاورقی انگلیسی)

\makeatletter
\def\@makeLTRfnmark{\hbox{\@textsuperscript{\latinfont\@thefnmark}}}
\renewcommand\@makefntext[1]{%
    \parindent 1em%
    \noindent
    \hb@xt@1.8em{\hss\if@RTL\@makefnmark\else\@makeLTRfnmark\fi}#1}
\makeatother

\renewcommand{\arraystretch}{1.2}
\begin{document}
\pagenumbering{harfi}
\thispagestyle{plain}
% !TeX root=main.tex
% در این فایل، عنوان پایان‌نامه، مشخصات خود، متن تقدیمی‌، ستایش، سپاس‌گزاری و چکیده پایان‌نامه را به فارسی، وارد کنید.
% توجه داشته باشید که جدول حاوی مشخصات پروژه/پایان‌نامه/رساله و همچنین، مشخصات داخل آن، به طور خودکار، درج می‌شود.
%%%%%%%%%%%%%%%%%%%%%%%%%%%%%%%%%%%%
% دانشگاه خود را وارد کنید
\university{علم و صنعت ایران}
% دانشکده، آموزشکده و یا پژوهشکده  خود را وارد کنید
\faculty{دانشکده مهندسی مکانیک}
% گروه آموزشی خود را وارد کنید
\department{گرایش دینامیک، ارتعاشات و کنترل}
% گروه آموزشی خود را وارد کنید
\subject{مهندسی مکانیک}
% گرایش خود را وارد کنید
\field{دینامیک، ارتعاشات و کنترل}
% چکیده عنوان پایان‌نامه را برای هدر وارد کنید
\newcommand{\Onva}{پایان‌نامه با استفاده از  IUST-Thesis} 

%عنوان پایان نامه را برای درج در روی جلد وارد کنید
\title{ نوشتن پروژه، پایان‌نامه و رساله
	 \\[5mm]
	   با استفاده از کلاس IUST-Thesis}
% نام استاد(ان) راهنما را وارد کنید
\firstsupervisor{دکتر محمود امین‌طوسی}
%\secondsupervisor{}
% نام استاد(دان) مشاور را وارد کنید. چنانچه استاد مشاور ندارید، دستور پایین را غیرفعال کنید.
\firstadvisor{استاد مشاور اول}
%\secondadvisor{استاد مشاور دوم}
% نام دانشجو را وارد کنید
\name{حسین}
% نام خانوادگی دانشجو را وارد کنید
\surname{بهبودی فام}
% شماره دانشجویی دانشجو را وارد کنید
\studentID{92741522}
% تاریخ پایان‌نامه را وارد کنید
\thesisdate{بهمن ۱۳۹4}
% به صورت پیش‌فرض برای پایان‌نامه‌های کارشناسی تا دکترا به ترتیب از عبارات «پروژه»، «پایان‌نامه» و »رساله» استفاده می‌شود؛ اگر  نمی‌پسندید هر عنوانی را که مایلید در دستور زیر قرار داده و آنرا از حالت توضیح خارج کنید.
%\projectLabel{پایان‌نامه}

% به صورت پیش‌فرض برای عناوین مقاطع تحصیلی کارشناسی تا دکترا به ترتیب از عبارات «کارشناسی»، «کارشناسی ارشد» و »دکترا» استفاده می‌شود؛ اگر  نمی‌پسندید هر عنوانی را که مایلید در دستور زیر قرار داده و آنرا از حالت توضیح خارج کنید.
%\degree{}
%\firstPage
\firstPage
\besmPage
\davaranPage

%\vspace{.5cm}
% در این قسمت اسامی اساتید راهنما، مشاور و داور باید به صورت دستی وارد شوند
%\renewcommand{\arraystretch}{1.2}
\begin{center}
\begin{tabular}{| p{8mm} | p{18mm} | p{.17\textwidth} |p{14mm}|p{.2\textwidth}|c|}
\hline
ردیف	& سمت & نام و نام خانوادگی & مرتبه \newline دانشگاهی &	دانشگاه یا مؤسسه &	امضـــــــــــــا\\
\hline
۱  &	استاد راهنما & دکتر \newline  محمود امین‌طوسی& استادیار & دانشگاه \newline حکیم سبزواری &  \\
\hline
۳ &      استاد مدعو\newline  خارجی			 & دکتر \newline منصور \newline نیکخواه بهرامی & استاد & دانشگاه \newline تهران  & \\
\hline
۷ &	استاد مدعو\newline  داخلی			 & دکتر \newline مرتضی منتظری & استاد& دانشگاه \newline  علم و صنعت ایران & \\
\hline
\end{tabular}
\end{center}
\esalatPage
\mojavezPage
% چنانچه مایل به چاپ صفحات «تقدیم»، «نیایش» و «سپاس‌گزاری» در خروجی نیستید، خط‌های زیر را با گذاشتن ٪  در ابتدای آنها غیرفعال کنید.
 % پایان‌نامه خود را تقدیم کنید!
 \newpage
 \thispagestyle{plain}
{\Large تقدیم به:}\\

ارواح پاک و مطهر جان باختگان فاجعه منا.

% سپاس‌گزاری
\begin{acknowledgementpage}
سپاس خداوندگار حکیم را که با لطف بی‌کران خود، آدمی را زیور عقل آراست.


در آغاز وظیفه‌  خود  می‌دانم از زحمات بی‌دریغ استاد  راهنمای خود،  جناب آقای دکتر خان میرزا، صمیمانه تشکر و  قدردانی کنم  که قطعاً بدون راهنمایی‌های ارزنده‌  ایشان، این مجموعه  به انجام  نمی‌رسید.

 در پایان، بوسه می‌زنم بر دستان خداوندگاران مهر و مهربانی، پدر و مادر عزیزم و بعد از خدا، ستایش می‌کنم وجود مقدس‌شان را و تشکر می‌کنم از همسر مهربان و عزیزم به پاس عاطفه سرشار و گرمای امیدبخش وجودش، که بهترین پشتیبان من بود.
% با استفاده از دستور زیر، امضای شما، به طور خودکار، درج می‌شود.
\signature 
\end{acknowledgementpage}
%%%%%%%%%%%%%%%%%%%%%%%%%%%%%%%%%%%%
% کلمات کلیدی پایان‌نامه را وارد کنید
\keywords{زی‌پرشین، لاتک، قالب پایان‌نامه، الگو}
%چکیده پایان‌نامه را وارد کنید، برای ایجاد پاراگراف جدید از \\ استفاده کنید. اگر خط خالی دشته باشید، خطا خواهید گرفت.
\fa-abstract{
این پایان‌نامه، به بحث در مورد نوشتن پروژه، پایان‌نامه و رساله با استفاده از کلاس 
\lr{IUST-Thesis}
می‌پردازد.
حروف‌چینی پروژه کارشناسی، پایان‌نامه یا رساله یکی از موارد پرکاربرد استفاده از زی‌پرشین است. 
زی‌پرشین بسته‌ای است که به همت آقای وفا خلیقی آماده شده است و امکان حروف‌چینی فارسی در \lr{\LaTeXe}{} را  برای فارسی‌زبانان فراهم کرده است.
از جمله مزایای لاتک آن است که در صورت وجود یک کلاس آماده برای حروف‌چینی یک سند خاص مانند یک پایان‌نامه، کاربر بدون درگیری با جزییات حروف‌چینی و صفحه‌آرایی می‌تواند سند خود را آماده نماید.
\\
شاید با قالب‌های لاتکی که برخی از مجلات برای مقالات خود عرضه می‌کنند مواجه شده باشید. اگر نظیر این کار در دانشگاههای مختلف برای اسناد متنوع آنها مانند پایا‌ن‌نامه‌ها آماده شود، دانشجویان به جای وقت گذاشتن روی صفحه‌آرایی مطالب خود، روی محتوای متن خود تمرکز خواهند نمود. به علاوه با آشنایی با لاتک خواهند توانست از امکانات بسیار این نرم‌افزار جهت نمایش بهتر دست‌آوردهای خود استفاده کنند.
به همین خاطر، یک کلاس با نام 
\lr{IUST-Thesis}
برای حروف‌چینی پروژه‌ها، پایان‌نامه‌ها و رساله‌های دانشگاه علم و صنعت ایران با استفاده از نرم‌افزار زی‌پرشین،  آماده شده است. این فایل به 
گونه‌ای طراحی شده است که کلیات خواسته‌های مورد نیاز  مدیریت تحصیلات تکمیلی دانشگاه علم و صنعت ایران را برآورده می‌کند و نیز، حروف‌چینی بسیاری از قسمت‌های آن، به طور خودکار انجام می‌شود.
}

\abstractPage
\thispagestyle{plain}
\newpage
\clearpage
\baselineskip=0.97cm
\pagestyle{plain}{
\tableofcontents \newpage
\listoffigures \newpage
\listoftables  \newpage
}
\baselineskip=0.97cm
\chapter*{فهرست علائم اختصاری}
\addcontentsline{toc}{chapter}{فهرست علائم اختصاری}

\persiangloss{جرم}{$m$ (kg)}
\persiangloss{جرم بی بعد شده}{$\hat{m}$}
\persiangloss{تانسور ممان اینرسی جرمی}{$I$ (kg.m$^2$)}
\persiangloss{تانسور ممان اینرسی جرمی بی بعد شده}{$\hat{I}$}
\persiangloss{بردار نیرو}{$\vec{F}$ (N)}
\persiangloss{بردار سرعت خطی}{$\vec{V}$ (m/s)}
\persiangloss{بردار سرعت زاویه‌ای}{$\vec{\omega}$ (rad/s)}
\persiangloss{بردار گشتاور}{$\vec{M}$ (N.m)}

\chapter*{قراردادهای نگارشی}
\addcontentsline{toc}{chapter}{قراردادهای نگارشی}

\persiangloss{اسکالر}{$a$}
\persiangloss{ماتریس}{$A$}
\persiangloss{کواترنین}{$\textbf{A,a}$}
\persiangloss{بردار}{$\vec{a},\vec{A}$}
\persiangloss{بردار سرعت خطی دستگاه $b$ نسبت به دستگاه $a$}{$\vec{V}_{b/a}$}
\persiangloss{نرمال شده کمیت $a$  }{$\hat{a}$}
\newpage


% اگر شما فصل اول  خود را در فایلی به جز chapter1 همراه با این کلاس نوشته‌اید باید چندخط اول chapter1 را در فایل خود کپی کنید.
\MyHeader
\pagestyle{fancy}
\pagenumbering{arabic}
\include{intro}			% فصل اول: مقدمه
\include{latexIntro}		% فصل دوم: آشنایی مقدماتی با لاتک
% مراجع
\MyBibHeader %هدر مراجع با هدر دیگر صفحات متفاوت است.
\bibliographystyle{acm-fa}%{chicago-fa}%{plainnat-fa}%
\bibliography{MyReferences}

\appendix                           %فصلهای پس از این قسمت به عنوان ضمیمه خواهند آمد.
% اگر شما پیوست اول  خود را در فایلی به جز appendix1 همراه با این کلاس نوشته‌اید باید چندخط اول appendix1 را در فایل خود کپی کنید.
\MyHeader
\pagestyle{fancy}
\include{appendix1}		% پیوست اول: مدیریت مراجع در لاتک
\include{appendix2}

\baselineskip=0.97cm

\printindex
% !TeX root=main.tex
% در این فایل، عنوان پایان‌نامه، مشخصات خود و چکیده پایان‌نامه را به انگلیسی، وارد کنید.

%%%%%%%%%%%%%%%%%%%%%%%%%%%%%%%%%%%%
\baselineskip=.6cm
\begin{latin}
\latinuniversity{Iran University of Science and Technology}
\latinfaculty{Mechanical Engineering Department}
\latinsubject{Mechanical Engineering }
\latinfield{Dynamic and Control}
\latintitle{Writing projects, theses and dissertations using IUST-Thesis Class}
\firstlatinsupervisor{Dr. Mahmood Amintoosi}
%\secondlatinsupervisor{Second Supervisor}
\latinname{Hossein}
\latinsurname{BehboodiFam}
\latinthesisdate{January 2016}
\latinkeywords{Writing Thesis, Template, \LaTeX, \XePersian}
\en-abstract{
	This thesis studies on writing projects, theses and dissertations using IUST-Thesis Class. It ...
}
\latinfirstPage
\end{latin}

\label{LastPage}

\end{document}