 % در این فایل، دستورها و تنظیمات مورد نیاز، آورده شده است.
%-------------------------------------------------------------------------------------------------------------------

% در ورژن جدید زی‌پرشین برای تایپ متن‌های ریاضی، این سه بسته، حتماً باید فراخوانی شود
\usepackage{amsthm,amssymb}
%گزینه [fleqn] برای چپ چین کردن تمام فرمولها است. (برای وسط چین بودن آن‌ها، آکولاد را حذف کنید.)
\usepackage[fleqn]{amsmath}
% بسته‌ای برای تنطیم حاشیه‌های بالا، پایین، چپ و راست صفحه
\usepackage[top=30mm, bottom=30mm, left=25mm, right=40mm]{geometry}
% بسته‌‌ای برای ظاهر شدن شکل‌ها و تصاویر متن
\usepackage{graphicx}
\usepackage{pgfplots}
\usepackage{tikz}
\usetikzlibrary{calc,intersections}
% بسته‌ای برای رسم کادر
\usepackage{framed} 
% بسته‌‌ای برای چاپ شدن خودکار تعداد صفحات در صفحه «معرفی پایان‌نامه»
\usepackage{lastpage}
% بسته‌ و دستوراتی برای ایجاد لینک‌های رنگی با امکان جهش
\usepackage[pagebackref=false,colorlinks,linkcolor=blue,citecolor=blue]{hyperref}
% چنانچه قصد پرینت گرفتن نوشته خود را دارید، خط بالا را غیرفعال و  از دستور زیر استفاده کنید چون در صورت استفاده از دستور زیر‌‌، 
% لینک‌ها به رنگ سیاه ظاهر خواهند شد که برای پرینت گرفتن، مناسب‌تر است
\usepackage{xcolor}
\hypersetup{
	colorlinks,
	linkcolor={black},
	citecolor={black},
	urlcolor={black}
}
% بسته‌ای برای تنظیم نحوه ظاهر شدن اولین صفحه فصلها
\usepackage{titlesec}
% بسته‌ لازم برای تنظیم سربرگ‌ها
\usepackage{fancyhdr}
% تنظیم سر برگ دو خطی (مخصوص دانشگاه علم و صنعت)
\renewcommand\headrule{%
  \begingroup
  \vspace{5pt}
  \hrule height 2pt width\headwidth
  \vspace{1pt}
  \hrule height 0.7pt width\headwidth
  \endgroup
}

\def\MyHeader{\titleformat{\chapter}[display]{\centering \normalfont\huge\bfseries}
{ {\chaptertitlename} { \thechapter:}}{20pt}{\Huge}[\newpage \thispagestyle{fancy}]}
\def\MyBibHeader{\titleformat{\chapter}[display]{\centering \normalfont\huge\bfseries}
{ {\chaptertitlename} { \thechapter:}}{20pt}{\Huge}[\newpage \thispagestyle{plain}]}
\def\onvan{عنوان}

\def\MATtextbaseline{1.5}
\renewcommand{\baselinestretch}{\MATtextbaseline}
%
\usepackage{setspace}
\usepackage{algorithm}
\usepackage{algorithmic}
\usepackage{subfigure}
\usepackage[subfigure]{tocloft}
% بسته‌ای برای ظاهر شدن «مراجع» و «نمایه» در فهرست مطالب
\usepackage[nottoc]{tocbibind}
\setlength{\cftfignumwidth}{12mm}
% دستورات مربوط به ایجاد نمایه
\usepackage{makeidx}
\makeindex
% فراخوانی بسته زی‌پرشین و تعریف قلم فارسی و انگلیسی
% در صورتی که میخاهید زی‌پرشین تصمیم بگیرد که حروف را کشیده کند گزینه [Kashida] را به زی‌پرشین اضافه کنید.
\usepackage{xepersian}
\settextfont[Scale=1]{HM XZar}
\setlatintextfont[Scale=0.9]{Times New Roman}

%%%%%%%%%%%%%%%%%%%%%%%%%%
% چنانچه می‌خواهید اعداد در فرمول‌ها، انگلیسی باشد، خط زیر را غیرفعال کنید. (توجه داشته باشید که فونت‌های زیر بر روی رایانه شما نصب شده باشد.)
%\setdigitfont[Scale=1]{Persian Modern}%{XB Zar}
%%%%%%%%%%%%%%%%%%%%%%%%%%
% تعریف قلم‌های فارسی و انگلیسی اضافی برای استفاده در بعضی از قسمت‌های متن
\defpersianfont\titlefont[Scale=1]{HM XTitr}
% \defpersianfont\iranic[Scale=1.1]{XB Zar Oblique}%Italic}%
% \defpersianfont\nastaliq[Scale=1.2]{IranNastaliq}

%%%%%%%%%%%%%%%%%%%%%%%%%%
% دستوری برای حذف کلمه «چکیده»
\renewcommand{\abstractname}{}
% دستوری برای حذف کلمه «abstract»
%\renewcommand{\latinabstract}{}
% دستوری برای تغییر نام کلمه «اثبات» به «برهان»
\renewcommand\proofname{\textbf{برهان}}
% دستوری برای تغییر نام کلمه «کتاب‌نامه» به «مراجع»
\renewcommand{\bibname}{مراجع}
% دستوری برای تعریف واژه‌نامه انگلیسی به فارسی
\newcommand\persiangloss[2]{#1\dotfill\lr{#2}\\}
% دستوری برای تعریف واژه‌نامه فارسی به انگلیسی 
\newcommand\englishgloss[2]{#2\dotfill\lr{#1}\\}
% تعریف دستور جدید «\پ» برای خلاصه‌نویسی جهت نوشتن عبارت «پروژه/پایان‌نامه/رساله»
\newcommand{\پ}{پروژه/پایان‌نامه/رساله }

%\newcommand\BackSlash{\char`\\}

%%%%%%%%%%%%%%%%%%%%%%%%%%

\SepMark{-}
\setlength{\cftsubsecnumwidth}{4em}
\setlength{\cftsecnumwidth}{3em}
% تعریف و نحوه ظاهر شدن عنوان قضیه‌ها، تعریف‌ها، مثال‌ها و ...
\theoremstyle{definition}
\newtheorem{definition}{تعریف}[section]
\theoremstyle{theorem}
\newtheorem{theorem}[definition]{قضیه}
\newtheorem{lemma}[definition]{لم}
\newtheorem{proposition}[definition]{گزاره}
\newtheorem{corollary}[definition]{نتیجه}
\newtheorem{remark}[definition]{ملاحظه}
\theoremstyle{definition}
\newtheorem{example}[definition]{مثال}

%\renewcommand{\theequation}{\thechapter-\arabic{equation}}
%\def\bibname{مراجع}
\numberwithin{algorithm}{chapter}
\def\listalgorithmname{فهرست الگوریتم‌ها}
\def\listfigurename{فهرست تصاویر}
\def\listtablename{فهرست جداول}

%%%%%%%%%%%%%%%%%%%%%%%%%%%%

%\doublespacing
%%%%%%%%%%%%%%%%%%%%%%%%%%%%%
% دستوراتی برای اضافه کردن کلمه «فصل» در فهرست مطالب

\newlength\mylenprt
\newlength\mylenchp
\newlength\mylenapp

\renewcommand\cftpartpresnum{\partname~}
\renewcommand\cftchappresnum{\chaptername~}
\renewcommand\cftchapaftersnum{:}

\settowidth\mylenprt{\cftpartfont\cftpartpresnum\cftpartaftersnum}
\settowidth\mylenchp{\cftchapfont\cftchappresnum\cftchapaftersnum}
\settowidth\mylenapp{\cftchapfont\appendixname~\cftchapaftersnum}
\addtolength\mylenprt{\cftpartnumwidth}
\addtolength\mylenchp{\cftchapnumwidth}
\addtolength\mylenapp{\cftchapnumwidth}

\setlength\cftpartnumwidth{\mylenprt}
\setlength\cftchapnumwidth{\mylenchp}	

\makeatletter
{\def\thebibliography#1{\chapter*{\refname\@mkboth
   {\uppercase{\refname}}{\uppercase{\refname}}}\list
   {[\arabic{enumi}]}{\settowidth\labelwidth{[#1]}
   \rightmargin\labelwidth
   \advance\rightmargin\labelsep
   \advance\rightmargin\bibindent
   \itemindent -\bibindent

   \listparindent \itemindent
   \parsep \z@
   \usecounter{enumi}}
   \def\newblock{}
   \sloppy
   \sfcode`\.=1000\relax}}
\makeatother

% تنظیم سفارشی سربرگ
\pagestyle{fancy}
\fancyhf{}
 \fancyhead[L]{ \leftmark}
 \fancyhead[R]{\Onva}
 \fancyfoot[C]{\thepage}
 \renewcommand{\chaptermark}[1]{%
\markboth{\ #1}{}}

% بسته ای برای ریست کردن شماره پاورقی در هر صفحه
\usepackage{perpage} %the perpage package
\MakePerPage{footnote}
%دستوراتی برای پاورقی (شماره گذاری انگلیسی برای پاورقی انگلیسی)

\makeatletter
\def\@makeLTRfnmark{\hbox{\@textsuperscript{\latinfont\@thefnmark}}}
\renewcommand\@makefntext[1]{%
    \parindent 1em%
    \noindent
    \hb@xt@1.8em{\hss\if@RTL\@makefnmark\else\@makeLTRfnmark\fi}#1}
\makeatother

\renewcommand{\arraystretch}{1.2}